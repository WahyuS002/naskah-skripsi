\chapter{TINJAUAN PUSTAKA}

\section{Karya Ilmiah}
Karya ilmiah, termasuk skripsi, memainkan peran krusial dalam pendidikan tinggi sebagai sarana pengembangan kemampuan penelitian dan pemikiran kritis mahasiswa. Hiteshue et al. \cite{hiteshue2015interdisciplinary} menekankan bahwa proses penyusunan karya ilmiah melibatkan berbagai tahapan kompleks, mulai dari pemilihan topik hingga presentasi hasil penelitian. Melalui proses ini, mahasiswa tidak hanya memenuhi syarat kelulusan, tetapi juga mengasah keterampilan esensial untuk karir masa depan mereka. Lapa-Asto et al. \cite{lapa2019impact} lebih lanjut menggarisbawahi pentingnya karya ilmiah dalam mengembangkan kemampuan analitis, pemecahan masalah, dan komunikasi ilmiah mahasiswa, yang sangat bernilai dalam dunia profesional.

\section{Machine Learning}
Machine learning, sebagai cabang kecerdasan buatan, memungkinkan sistem komputer untuk belajar dan meningkatkan kinerja dari pengalaman tanpa pemrograman eksplisit. Ray \cite{ray2019quick} menjelaskan bahwa pendekatan machine learning umumnya terbagi menjadi beberapa kategori utama, termasuk pembelajaran terbimbing, tak terbimbing, semi-terbimbing, dan pembelajaran penguatan. Setiap pendekatan ini memiliki karakteristik dan aplikasi yang berbeda, memungkinkan penggunaan machine learning dalam berbagai bidang. Gupta dan Roman-Gonzalez \cite{gupta2020survey} memperluas pemahaman ini dengan menunjukkan aplikasi luas machine learning, mulai dari pengenalan pola hingga analisis prediktif, yang mencerminkan fleksibilitas dan kekuatan teknologi ini dalam menangani berbagai jenis data dan permasalahan.

\section{Deep Learning}
Deep learning, subbidang machine learning yang berbasis jaringan saraf tiruan dengan pembelajaran representasi, telah menunjukkan keberhasilan luar biasa dalam berbagai aplikasi. Dalam konteks sistem rekomendasi, Zarzour et al. \cite{zarzour2019recdnning} mendemonstrasikan bagaimana pendekatan deep learning dapat menangkap interaksi kompleks antara pengguna dan item pada tingkat yang lebih dalam, melampaui kemampuan teknik tradisional seperti faktorisasi matriks. Keunggulan ini dipertegas oleh Almaghrabi dan Chetty \cite{almaghrabi2018deep}, yang menunjukkan bahwa kemampuan deep learning untuk mempelajari fitur laten secara otomatis dari data mentah dapat secara signifikan meningkatkan kinerja sistem rekomendasi, terutama dalam hal akurasi prediksi dan personalisasi.

\section{Sistem Rekomendasi}
Di era informasi digital saat ini, sistem rekomendasi menjadi semakin penting dalam membantu pengguna menavigasi banjir data yang tersedia. Mana et al. \cite{mana2021machine} menjelaskan bagaimana sistem ini dirancang untuk menganalisis perilaku pengguna, preferensi, dan karakteristik item untuk memprediksi dan menyarankan konten yang relevan. Pendekatan ini tidak hanya membantu mengatasi masalah kelebihan informasi tetapi juga meningkatkan pengalaman pengguna dalam berbagai domain. Yu et al. \cite{yu2021collaborative} lebih lanjut mengeksplorasi bagaimana penerapan teknik machine learning, seperti collaborative filtering dan content-based filtering, telah meningkatkan keakuratan dan efektivitas sistem rekomendasi dalam memberikan saran yang dipersonalisasi, menunjukkan evolusi dan sofistikasi terus-menerus dalam bidang ini.

\section{Hybrid Modeling}
Hybrid modeling muncul sebagai pendekatan yang menggabungkan kekuatan berbagai teknik pemodelan untuk mengatasi keterbatasan metode tunggal. Polyiam dan Boonrawd \cite{polyiam2017hybrid} mendemonstrasikan efektivitas pendekatan hybrid dalam konteks peramalan harga, menggabungkan jaringan saraf tiruan dengan teknik support vector machine untuk meningkatkan akurasi prediksi. Dalam ranah yang berbeda, Zhu et al. \cite{zhu2018hybrid} mengaplikasikan model hybrid deep learning untuk penilaian kredit konsumen, menunjukkan bagaimana integrasi berbagai algoritma dapat menghasilkan performa yang lebih baik dibandingkan metode tunggal. Kedua studi ini menegaskan potensi hybrid modeling dalam menangkap pola dan hubungan yang lebih kompleks dalam data, serta meningkatkan ketahanan dan fleksibilitas sistem dalam menghadapi berbagai jenis data dan skenario penggunaan.
