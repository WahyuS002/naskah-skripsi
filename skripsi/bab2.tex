\chapter{TINJAUAN PUSTAKA}

\section{Lama Studi Mahasiswa dan Dampaknya}
Lama studi mahasiswa merupakan indikator penting yang mencerminkan kualitas dan efektivitas pendidikan tinggi, serta memiliki dampak signifikan baik bagi mahasiswa maupun institusi. Ketepatan waktu kelulusan tidak hanya menjadi perhatian utama bagi penyelenggara pendidikan tinggi dalam merancang kegiatan pengembangan yang bermakna, tetapi juga menjadi salah satu kriteria penilaian akreditasi oleh Badan Akreditasi Nasional Perguruan Tinggi (BAN-PT) \cite{wirawan2019application}. Keterlambatan kelulusan dapat mengakibatkan berbagai konsekuensi negatif, seperti penurunan efisiensi sistem pendidikan, penumpukan mahasiswa, serta implikasi ekonomi dan sosial bagi mahasiswa dan masyarakat luas. Hal ini dipertegas oleh temuan bahwa tingkat kelulusan tepat waktu di beberapa perguruan tinggi masih berada di bawah 50\%, menunjukkan adanya kesenjangan antara harapan dan realitas \cite{dengen2018student}. Oleh karena itu, pemahaman mendalam tentang faktor-faktor yang mempengaruhi lama studi mahasiswa dan pengembangan strategi untuk meningkatkan tingkat kelulusan tepat waktu menjadi sangat penting bagi institusi pendidikan tinggi dalam upaya meningkatkan kualitas pendidikan dan mempertahankan akreditasi yang baik.

\section{Machine Learning}
Machine learning, sebagai cabang kecerdasan buatan, memungkinkan sistem komputer untuk belajar dan meningkatkan kinerja dari pengalaman tanpa pemrograman eksplisit. Ray \cite{ray2019quick} menjelaskan bahwa pendekatan machine learning umumnya terbagi menjadi beberapa kategori utama, termasuk pembelajaran terbimbing, tak terbimbing, semi-terbimbing, dan pembelajaran penguatan. Setiap pendekatan ini memiliki karakteristik dan aplikasi yang berbeda, memungkinkan penggunaan machine learning dalam berbagai bidang. Gupta dan Roman-Gonzalez \cite{gupta2020survey} memperluas pemahaman ini dengan menunjukkan aplikasi luas machine learning, mulai dari pengenalan pola hingga analisis prediktif, yang mencerminkan fleksibilitas dan kekuatan teknologi ini dalam menangani berbagai jenis data dan permasalahan.

\section{Jaringan Syaraf Tiruan}
Jaringan Syaraf Tiruan (JST) adalah sistem komputasi yang terinspirasi dari prinsip kerja jaringan syaraf biologis otak manusia. JST terdiri dari sejumlah besar unit pemrosesan sederhana yang saling terhubung, disebut neuron artifisial, yang bekerja secara paralel untuk memproses informasi dan menghasilkan output berdasarkan input yang diterima. Setiap koneksi antar neuron memiliki bobot yang dapat dimodifikasi melalui proses pembelajaran, memungkinkan JST untuk beradaptasi dan mempelajari pola dari data yang diberikan. Arsitektur dasar JST terdiri dari lapisan input, satu atau lebih lapisan tersembunyi, dan lapisan output, dengan setiap neuron memiliki fungsi aktivasi yang menentukan outputnya berdasarkan input yang diterima. Salah satu keunggulan utama JST adalah kemampuannya untuk melakukan generalisasi dan menangani masalah non-linear, membuatnya cocok untuk berbagai aplikasi seperti pengenalan pola, klasifikasi, prediksi, dan pengambilan keputusan. Implementasi digital JST, terutama menggunakan FPGA, menawarkan keuntungan dalam hal kecepatan komputasi dan kemampuan rekonfigurasi, meskipun memerlukan pertimbangan khusus dalam hal representasi bobot, implementasi fungsi aktivasi, dan algoritma pembelajaran seperti backpropagation \cite{amrutha2018performance}.

\section{Sistem Prediksi}
Sistem prediksi adalah suatu pendekatan komputasional yang dirancang untuk memperkirakan atau meramalkan hasil atau kejadian di masa depan berdasarkan analisis data historis dan pola yang ada. Sistem ini menggunakan berbagai teknik dan algoritma, termasuk metode statistik, machine learning, dan kecerdasan buatan, untuk mengidentifikasi tren, pola, dan hubungan dalam data yang dapat digunakan untuk membuat perkiraan yang akurat \cite{dengen2018student}. Penerapan sistem prediksi mencakup berbagai bidang, mulai dari peramalan cuaca dan analisis pasar keuangan hingga prediksi perilaku konsumen dan hasil akademik mahasiswa. Dalam konteks pendidikan tinggi, sistem prediksi dapat digunakan untuk mengidentifikasi mahasiswa yang berisiko mengalami keterlambatan kelulusan atau putus kuliah, memungkinkan institusi untuk mengambil tindakan preventif dan memberikan dukungan yang diperlukan \cite{wirawan2019application}. Keakuratan dan kehandalan sistem prediksi sangat bergantung pada kualitas data yang digunakan, pemilihan fitur yang relevan, dan kesesuaian algoritma yang diterapkan dengan karakteristik masalah yang dihadapi.
